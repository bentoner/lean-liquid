\documentclass[english]{amsart}

\usepackage{amsmath}
\usepackage{amssymb}

\newcommand{\N}{\mathbb{N}}
\newcommand{\Q}{\mathbb{Q}}
\newcommand{\R}{\mathbb{R}}
\newcommand{\Z}{\mathbb{Z}}

\newcommand{\nnreal}{\mathbb{R}_{\geq 0}}

\DeclareMathOperator{\Hom}{Hom}

\usepackage{amsthm}

\newtheorem{theorem}{Theorem}


\begin{document}
\title[]{Blueprint for Gordan's Lemma}

\begin{abstract}
A blueprint for Gordan's Lemma.
\end{abstract}

\maketitle

What we are going for is the following. Let $\Lambda$ be a finite free $\Z$-module, and let $\Lambda^*:=\Hom(\Lambda,\Z)$ be its dual. If $S$ is a subset of $\Lambda$ (always finite, in practice, I think) then its \emph{dual cone} $S^\vee$ consists of the $t\in\Lambda^*$ such that $\langle s,t\rangle\geq0$ for all $s\in S$. Our goal is:

\begin{theorem}If $S$ is finite then $S^\vee$ is a finitely-generated additive monoid.
\end{theorem}

\begin{definition}
A \emph{cone} in a real vector space~$V$ is convex subset that is closed under multiplication by
non-negative reals.
\end{definition}

Equivalently, a cone is a $\nnreal$-subsemimodule of a real vector space.

We will be only interested in finite-dimensional vector spaces. For the most part, we will also be
only interested in \emph{closed} cones, that is, cones that are closed with respect to the
underlying topological space structure.





  \begin{definition}
A face of a
\end{definition}



\section{Stuff in the {\tt toric} branch I don't understand}

``The {\tt pregenerators} are elements of the dual set of $S$ that generate a 1-dimensional subcone of the dual set.  They should be exactly the extremal rays and any generating set of the dual set of $S$ should contain them.''

Q) what is a 1-dimensional subcone (and how can an element of $S^\vee$ \emph{not} generate a 1-dimensional subcone?). What is the definition of ``extremal ray''?

``Implicit in this definition is that we only consider subsets $T\subseteq S$ that produce a ray (a 1-dimensional subcone) contained in the dual cone of $S$.  Thus, not all maximal independent subsets `t ⊆ s` give rise to a {\tt pregenerator}.''

Q) What does ``produce'' mean here?

``Besides the pregenerators, we will have to "fill in" the holes.  Here is an example:''

The example is: $\Lambda=\Z^2$, $\Lambda^*$ is identified with $\Lambda$ via the dot product, $S=\{(1,0),(-1,2)\}$ and $S^\vee$ is then the additive monoid generated by $\{(0,1),(1,1),(2,1)\}$. It is claimed that $(0,1)$ and $(2,1)$ are primitive and it is observed that $(1,1)$ is not in their $\N$-span.

I can't make sense of the definition of {\tt pre\_generators} in the file. The dual set of a set which is closed under $s\mapsto -s$ is surely always a $\Z$-module and can hence never be generated as an $\N$-module by one element.

``If we start with a finite set $S$ of elements in $\Lambda$, then the set of pregenerators of $S^\vee$ is also finite.''

There is a long proof of this in {\tt gordan\_blueprint} but right now I don't understand the definition of {\tt pre\_generators} (I am confused about how anything non-zero can ever be a pregenerator).

In the file, $\Lambda^*$ is denoted $N$, and there is now some talk about some elements of $\Lambda^*\otimes\Q$ (which unfortunately also seems to be named $N$) which are in the $[0,1]\cap\Q$-span of some given set (also called $S$). If $S$ is finite then the elements of this form which live in a lattice should be finite, although this is not yet proved. There is a suggestion to go via $\R$, which I am very keen to avoid.

A {\tt full\_on} condition is introduced on a pairing which will be satisfied for our perfect pairing.

A special case of Gordan's Lemma is then stated with no proof, and the full proof is then claimed to follow from this.

Is someone able to fill in some of the mathematical details here?
\end{document}
