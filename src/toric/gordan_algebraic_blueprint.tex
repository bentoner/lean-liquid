\documentclass[english]{amsart}

\usepackage{amsmath}
\usepackage{amssymb}

\newcommand{\C}{\mathbb{C}}
\newcommand{\N}{\mathbb{N}}
\newcommand{\Q}{\mathbb{Q}}
\newcommand{\R}{\mathbb{R}}
\newcommand{\Z}{\mathbb{Z}}

\DeclareMathOperator{\Hom}{Hom}

\usepackage{amsthm}

\newtheorem{theorem}{Theorem}


\begin{document}
\title[]{Blueprint for Gordan's Lemma}

\begin{abstract}
A blueprint for Gordan's Lemma.
\end{abstract}

\maketitle

What we are going for is the following. Let $\Lambda$ be a finite free $\Z$-module, and let $\Lambda^*:=\Hom(\Lambda,\Z)$ be its dual. If $S$ is a subset of $\Lambda$ (always finite, in practice, I think) then its \emph{dual $\Z$-cone} $S^\vee$ consists of the $t\in\Lambda^*$ such that $\langle s,t\rangle\geq0$ for all $s\in S$. Our goal is:

\begin{theorem}If $S$ is finite then $S^\vee$ is a finitely-generated additive monoid.
\end{theorem}

\section{Algebraic proof}

We follow the algebraic proof in Wikipedia.

\begin{theorem} If $M$ is an additive abelian monoid and $R$ is a nonzero commutative ring, then $M$ is finitely-generated as a monoid iff the monoid algebra $R[M]$ is finitely-gnerated as an $R$-algebra.
\end{theorem}
\begin{proof} Because $R$ is nonzero, we can think of $M$ as a subset of $R[M]$.
  
  First say $M$ is finitely generated by $S\subseteq M$. The sub-$R$-algebra of $R[M]$ generated by $S$ contains all of $M$, and is an $R$-module so it's all of $R[M]$.

  Conversely, say $R[M]$ is finitely-generated. If $f\in R[M]$ then it has a support, which is a finite subset of~$M$. Now $f$ is in the sub-$R$-module generated by its support, and hence is in the sub-$R$-algebra generated by the support. So if we have finitely many $f$'s which generate $R[M]$ as an $R$-algebra then we can replace each $f$ by its support and get a finite subset of $M$ which generates a subalgebra of $R[M]$ which contains all the $f$'s and hence is all of $R[M]$. We deduce that a finite subset $S$ of $M$ generates $R[M]$ as $R$-algebra. We claim that $S$ generates $M$ as an additive monoid. This follows because the $R$-algebra generated by $S$ is the $R$-module generated by the monoid generated by $S$, and the monoid generated by $S$ is a subset of $M$.
\end{proof}

We prove Gordan's Lemma by induction on (the size of)~$S$. For $S$ empty the result is clear. For the inductive step, we need to check that if $S^\vee$ is finitely-generated then so is $(S\cup{v})^\vee$, which is $S^\vee\cap\{\phi\in\Lambda^*\,:\,\phi(v)\geq0,\}$. Set $M=S^\vee$ and write $A=\C[M]$; this is finitely-generated as a $\mathbb{C}$-algebra by the previous lemma. Define $f:M\to\Z$ by $f(\phi)=\phi(v)$. Define $A_n$ to be the $\C$-module generated by the $\phi\in M$ with $f(\phi)=n$; this puts a $\Z$-grading on $A$. By the previous lemma we need to prove that the subring $A_{\geq0}:=\oplus_{n\geq0}A_n$ is finitely-generated as a $\mathbb{C}$-algebra.

First note that $A_0=\C[T]$ where $T=\{\phi\in M\,:\,f(\phi)=0\}$ is a subalgebra, so it suffices to prove that $A_0$ is a finitely-generated $\C$-algebra and that $A_{\geq0}$ is a finitely-generated $A_0$-algebra.

{\bf TODO: prove these two facts.} Wikipedia claims there's a surjection $M\to T$ which would prove that $T$ is finitely-generated and thus that $A_0$ is finitely-generated as a $\C$-algebra, but I don't see it.

For the proof that $A_{\geq0}$ is a finitely-generated $A_0$-algebra, the Wikipedia article seems to use a variant of the proof of the Hilbert basis theorem and I've not read the details yet.

\end{document}
