\documentclass{amsart}

\usepackage{amsmath}
\usepackage{amssymb}

\newcommand{\C}{\mathbb{C}}
\newcommand{\N}{\mathbb{N}}
\newcommand{\Q}{\mathbb{Q}}
\newcommand{\R}{\mathbb{R}}
\newcommand{\Z}{\mathbb{Z}}

\newcommand{\Az}{A_{\geq0}}
\newcommand{\Ap}{A_{>0}}
\newcommand{\An}{A_{0\leq N}}
\newcommand{\ev}{\textrm{deg}_\varphi}

\newcommand{\ddt}[1]{\marginpar{{\tiny{DT: #1}}}}

\newcommand{\form}[1]{\medskip
\parbox{330pt}
{\small{{\textbf{Note for formalization.}}
#1}}
\medskip}

\DeclareMathOperator{\Hom}{Hom}
\DeclareMathOperator{\rk}{rk\,}

\usepackage{amsthm}

\newtheorem{theorem}{Theorem}
\newtheorem{lemma}[theorem]{Lemma}


\begin{document}
\title[]{Blueprint for Gordan's Lemma}

\begin{abstract}
A blueprint for Gordan's Lemma.
\end{abstract}

\maketitle

What we are going for is the following. Let $\Lambda$ be a finite free $\Z$-module, and let $\Lambda^*:=\Hom(\Lambda,\Z)$ be its dual. If $S$ is a subset of $\Lambda^*$ (always finite, in practice, I think) then its \emph{dual $\Z$-cone} $S^\vee$ consists of the $t\in\Lambda$ such that $\langle s,t\rangle\geq0$ for all $s\in S$.

Note that the dual $\Z$-cone is an additive submonoid of~$\Lambda$, but it need not be a $\Z$-module: multiplying by negative integers will reverse all the inequalities implied by being in the dual $\Z$-cone!  For instance $\N \subset \Z$ is the dual $\Z$-cone of $\{1\} \subset \Z$, viewing $\Z$ as its own dual, via multiplication.

Our goal is to prove the following result.

\begin{theorem}[Gordan's Lemma]
If $S \subset \Lambda^*$ is finite, then $S^\vee \subset \Lambda$ is a finitely-generated additive monoid.
\end{theorem}

\section{Algebraic proof}

We follow the algebraic proof in Wikipedia.

\begin{theorem} \label{t:RMfg}
If $M$ is an additive abelian monoid and $R$ is a nonzero commutative ring, then $M$ is finitely-generated as a monoid iff the monoid algebra $R[M]$ is finitely-generated as an $R$-algebra.
\end{theorem}
\begin{proof} Because $R$ is nonzero, we can think of~$M$ as a subset of $R[M]$: even more, we can think of~$M$ as a multiplicative submonoid of $R[M]$.

First, say~$M$ is generated by $S\subseteq M$. The $R$-subalgebra of $R[M]$ generated by~$S$ contains all of~$M$, and is an $R$-module so it's all of $R[M]$.  If~$S$ is finite, then $R[M]$ is finitely generated.

Conversely, say $R[M]$ is finitely-generated. If $f\in R[M]$ then it has a support, which is a finite subset of~$M$. Now $f$ is in the $R$-submodule generated by its support, and hence is in the $R$-subalgebra generated by the support. So if we have finitely many $f$'s which generate $R[M]$ as an $R$-algebra then we can replace each $f$ by its support and get a finite subset of $M$ which generates a subalgebra of $R[M]$ which contains all the $f$'s and hence is all of $R[M]$. We deduce that a finite subset $S$ of $M$ generates $R[M]$ as an $R$-algebra. We claim that $S$ generates $M$ as an additive monoid. This follows because the $R$-algebra generated by $S$ is the $R$-module generated by the monoid generated by $S$, and the monoid generated by $S$ is a subset of $M$.
\end{proof}

We prove Gordan's Lemma by two layers of induction.  First, proceed by induction on the rank of~$\Lambda$.

\noindent
{\textbf{First induction on $\rk \Lambda$: base case.}}
The result is clear in the case in which the rank of~$\Lambda$ is~$0$: in this case, $\Lambda = 0$ and the empty set generates the unique additive submodule of $\Lambda = 0$.

\noindent
{\textbf{First induction on $\rk \Lambda$: inductive step.}}
Assume that the rank of~$\Lambda$ is strictly positive and that, for every free, finitely generated $\Z$-module $\Lambda'$ of rank strictly smaller than the rank of~$\Lambda$, the dual set of every finite subset of $\left(\Lambda'\right)^*$ is a finitely generated additive monoid.

\form{We really only need the case of rank one less.  In fact, we are going to apply the inductive hypothesis to a submodule of~$\Lambda$ obtained as the kernel of a non-zero linear map $\Lambda \to \Z$.}

Proceed by induction on (the size of)~$S$ (within the inductive step of the first induction on $\rk \Lambda$).

\form{In the second induction we play around with~$S$.  Note that the set $S^\vee$ coincides with the dual $\Z$-cone on the set of points of the $\N$-submodule spanned by~$S$ (or even its saturation).  I, DT, do not know whether or not this observation makes the formalization simpler.}

\noindent
{\textbf{Second induction on $\# S$: base case.}}
For~$S$ empty the result is clear: the dual of the empty set is the whole~$\Lambda$ and if $\lambda \subset \Lambda$ generates~$\Lambda$ as a $\Z$-module, then $\lambda \cup \{ -\ell \, : \, \ell \in \lambda \}$ generates~$\Lambda$ as an additive monoid.

\noindent
{\textbf{Second induction on $\# S$: induction step.}}
For the inductive (in the size of $\#S$) step, it suffices to check that if $S^\vee$ is finitely-generated then so is $(S\cup{\varphi})^\vee$.  We use the equality
\[
(S\cup{\varphi})^\vee = S^\vee \cap \{v \in \Lambda \,:\, \varphi(v)\geq0\,\},
\]
which follows from the definitions.

The result is clear if $\varphi = 0$: in this case $\varphi$ imposes no extra condition on $S^\vee$, the equality
\[
(S \cup \{0\})^\vee = S^\vee
\]
holds, and we know the result for $S^\vee$.

Thus, assume that~$\varphi$ is non-zero.  Choose any non-zero, commutative ring with identity~$R$.  Set $M=S^\vee$ and write $A=R[M]$; this is finitely-generated as an $R$-algebra by Theorem~\ref{t:RMfg}. Define
\[
\ev \colon M\to\Z
\]
by $\ev (v)=\varphi(v)$. Define $A_n$ to be the $R$-module generated by the $v \in M$ with $\ev(v)=n$; this determines a $\Z$-grading on~$A$. By Theorem~\ref{t:RMfg}, it suffices to prove that the subring $A_{\geq0}:=\oplus_{n\geq0}A_n$ is finitely-generated as an $R$-algebra.

First note that $A_0=R[T]$ where $T=\{v \in M \,:\, \deg(v)=0\}$ is a subalgebra, so it suffices to prove that
\begin{itemize}
\item
$A_0$ is a finitely-generated $R$-algebra, and that
\item
$\Az$ is a finitely-generated $A_0$-algebra.
\end{itemize}

\begin{lemma}
The $R$-algebra $A_0 = R[T]$ is finitely generated.
\end{lemma}
\begin{proof}
We use the equivalence of Theorem~\ref{t:RMfg}: it suffices to show that~$T$ is finitely generated as a monoid.  Recall that, by definition,~$T$ is the submonoid of~$\Lambda$ satisfying
\[
T=\{v \in M \,:\, \deg(v) = \varphi (v) = 0\} \subset \ker \varphi .
\]
Since we reduced to the case in which~$\varphi$ is non-zero, we know that $\ker \varphi$ is a free, finitely-generated $\Z$-module of rank equal to ${\rk} \Lambda - 1$.

To apply the induction hypothesis, we check that~$T \subset \ker \varphi$ is the dual of a finite subset of $(\ker \varphi)^*$.  Observe that the dual of $\ker \varphi$ is the quotient of $\Lambda^*$ by the saturation of the additive subgroup generated by~$\varphi$.  By construction,~$T$ is therefore the dual set of the image of~$S$ under the projection
\[
\Lambda^* \to \left( \Lambda^* / \langle \varphi \rangle^{\textrm{sat}}  \right) \simeq \left( \ker \varphi \right)^*.
\]
By the induction step of the first induction (on the number of generators of~$\Lambda$), we know that~$T$ is finitely generated, as needed.
\end{proof}


\form{The saturation can be avoided by working, more generally, not with the dual of~$\Lambda$, but with a $\Z$-module of linear functionals on~$\Lambda$ that surjects onto the dual of~$\Lambda$.  Alternatively, it can also be avoided by replacing $\varphi$ by $\varphi' \in \Lambda^*$, where $\varphi = a \varphi'$, with $a \in \N$ chosen as the largest it can be for such an identity to hold.}

To prove the second, and final, step, we show the following more general result.

\begin{lemma}
Let~$A$ be a Noetherian $\Z$-graded ring.  Denote by $\Az = \oplus_{n \geq 0} A_n$ the sub-algebra of~$A$ consisting of the elements of~$A$ of non-negative degree.  The ring $\Az$ is finitely generated as an $A_0$-algebra.
\end{lemma}

\begin{proof}
Let~$I$ be the ideal of~$A$ that is generated by all the homogeneous elements of strictly positive degree.  (Note that, since~$A$ might have elements of negative degree, the ideal~$I$ might contain elements of negative degree as well.)

Since~$A$ is Noetherian, the ideal~$I$ admits a finite generating set: choose one and denote its elements by $f_1, \ldots , f_n$.  Since each element of~$I$ is an $A$-linear combination of homogeneous elements of strictly positive degree, we can replace each chosen generator by the collection of all the elements of~$I$ that appear in such linear combinations.  Thus, we further assume that the chosen generators are
\begin{itemize}
\item
homogeneous, and
\item
have strictly positive degree.
\end{itemize}
Let $N \in \N$ be the maximum of the degrees of the generators $f_1, \ldots , f_r$:
\[
N = \max \{ \deg f_1 , \ldots , \deg f_r \}.
\]
Let $\An \subset \Az$ be the subset consisting all the homogeneous elements of degree at most~$N$.  Note that, in particular, all the chosen generators $f_1, \ldots , f_r$ are contained in~$\An$.  We show that~$\An$ generates~$\Az$ as an $A_0$-algebra.

More precisely, we show that, for all $n \in \N$, every element $f \in \Az$ of degree~$n$ in the $A_0$-algebra $\Az$ is generated by~$\An$ as an $A_0$-algebra. (If this is any help, this step is entirely analogous to the proof that the Weak Mordell-Weil Theorem implies the Mordell-Weil Theorem: it is a relatively standard "Noetherian induction" argument.)

Proceed by induction on~$n$, starting the induction at $n = N$.  For the base case there is nothing to prove: the result is true if $n = N$, by definition of~$\An$.

Suppose that~$\An$ generates every element of~$\Az$ of degree at most~$n$, for some natural number~$n$ satisfying $N \le n$.  Let~$f$ be an element of~$\Az$ of degree $n+1$.  By homogeneity of the ideal, we can assume that~$f$ is homogeneous of degree~$n+1$.

Since $f_1, \ldots, f_r$ generate~$I$, the homogeneous element~$f$ admits a decomposition
\[
f = \sum_{i = 1}^r g_i f_i
\]
with $g_1, \ldots, g_r$ homogeneous elements.  Since the degrees of the generators $f_1, \ldots , f_r$ are strictly positive, the inequalities
\[
\deg g_1 < \deg f, \ldots , \deg g_r < \deg f
\]
hold.  Since the degree of~$f$ is~$n+1$ and~$n$ satisfies $N = \max \{\deg f_1 , \ldots , \deg f_r \} \le n$, the degrees of $g_1, \ldots , g_r$ satisfy
\[
0 \le \deg g_1 < \deg f, \ldots , 0 \le \deg g_r < \deg f.
\]
By the inductive hypothesis, each one of the elements $g_1, \ldots , g_r$ is in the $A_0$-algebra generated by~$\An$, as stated.

Thus, it suffices to show that~$\An$ is finitely generated as an $A_0$-module.  For this, we show that, for each natural number~$n$, the homogeneous degree piece~$A_n$ is finitely generated as an $A_0$-module.

\form{For the given proof, it seems important that we work with a {\emph{unique}} graded piece: I do not see right away how to make the argument work with~$\An$ directly. }

This is again a consequence of Noetherianity of~$A$.  Suppose that
\[
N_1 \subset N_2 \subset \cdots \subset N_i \subset \cdots
\]
is an increasing chain of $A_0$-submodules of $A_n$, such that $\cup_i N_i = A_n$.  The chain of ideals
\[
N_1 A \subset N_2 A \subset \cdots \subset N_i A \subset \cdots
\]
stabilizes, since~$A$ is Noetherian.  Intersecting with $A_n$, we find that the sequence
\[
N_1 A \cap A_n \subset N_2 A \subset \cap A_n \cdots \subset N_i A \cap A_n \subset \cdots
\]
also stabilizes.  Finally, we observe that, for all indices~$i$, the equality $N_i A \cap A_n = N_i$ holds: since all the elements of~$N_i$ are homogeneous of degree~$n$, the only $A$-multiples of the elements of~$N_i$ that have degree~$n$ are the multiples by homogeneous elements of degree~$0$.  Since~$N_i$ is an $A_0$-module, we are done.
\end{proof}
\end{document}

Proceed by induction on the degree of the elem

Second, let~$f$ be a homogeneous element of~$A_z$ of degree~$N$.  we proceed by induction on the degree~$N$ of the

%Let $N \ge N_0$ a natural number.
Since $f_1, \ldots, f_n$ generate~$I$, each homogeneous element~$f$ admits a decomposition
\[
f = \sum_{i = 1}^n g_i f_i
\]
with $g_1, \ldots, g_n$ homogeneous elements.  If the degree of~$f$ is at least~$N_0$ the degrees of $g_1, \ldots , g_n$ satisfy
\[
0 \le \deg g_1 < N, \ldots , 0 \le \deg g_n < \deg f = N.
\]

Let~$f$ be a

We deduce that the $A_0$-algebra $\Az$ is generated by $f_1, \ldots , f_n$ together with all the homogeneous elements of degree at most


%{\bf TODO: prove these two facts.} Wikipedia claims there's a surjection $M\to T$ which would prove that $T$ is finitely-generated and thus that $A_0$ is finitely-generated as a $\C$-algebra, but I don't see it.
% DT: I thought a bit about the argument that the Wikipedia page says, but I could not come up with a projection.  However, I think that the one above works.

For the proof that $A_{\geq0}$ is a finitely-generated $A_0$-algebra, the Wikipedia article seems to use a variant of the proof of the Hilbert basis theorem and I've not read the details yet.

\end{document}


\begin{lemma}
Let $S \subset \Lambda^*$ be a finite set and let $v \in \Lambda^*$ be a non-zero element.  The $\C$-algebra $A_0 = \C[\ker v]$ is finitely generated.
\end{lemma}
